%%%%%%%%%%%%%%%%%%%%%%%%%%%%%%%%%%%%%%%%%%%%%%%%%%%%%%%%%%%%%%%%%%% 
%% Conclusiones 
%%%%%%%%%%%%%%%%%%%%%%%%%%%%%%%%%%%%%%%%%%%%%%%%%%%%%%%%%%%%%%%%%%%



\section{ Conclusiones}
\begin{frame}
  \frametitle{\textbf{Tabla de Contenidos}}
  \begin{center}
    {\vspace{-1.5cm}\Large \textbf{Sección \thesection}\vspace{0.5cm}}
    \begin{beamercolorbox}[
      sep=8pt,center]{part title}
      \usebeamerfont{part title}
      \textbf{\insertsection}
    \end{beamercolorbox}
  \end{center}
\end{frame}


\begin{frame}
  \frametitle{\textbf{\textbf{Conclusiones}}}
    \begin{block}{\centering \textbf{Técnica `probabilistic shaping'}}
    \begin{itemize}\small
        \item Permite disminuir la tasa de error total del sistema.
        \item Agrega una determinada cantidad de bits de redundancia.
        \item Se vuelve mas eficiente a medida que la longitud de entrada aumenta.
    \end{itemize}
    \end{block}
    \vspace{-0.2cm}

    \begin{block}{\centering \textbf{Técnica `constant distribution matching'}}
       \begin{itemize}\small
        \item Se debe utilizar técnica de escalado.
        \item La actualización de la nueva probabilidad implica realizar una división.
        \item Fácil adaptación a nuevas distribuciones de probabilidad. 
    \end{itemize}
    \end{block}
    \vspace{-0.2cm}

    \begin{block}{\centering \textbf{Implementación}}
    \begin{itemize}\small
        \item El calculo de los intervalos, se puede realizar de manera recursiva o secuencial.
        \item  El sistema implementado es poco practico para una longitud de entrada de 4 bits. 
    \end{itemize}
    \end{block}
\end{frame}