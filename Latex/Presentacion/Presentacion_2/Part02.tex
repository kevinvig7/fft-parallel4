\section{Design of FFT architecture via folding transformation}
\subsection{Parallel radix-$2^3$ 16-Points}
\begin{frame}
  \frametitle{\textbf{Table of Contents}}
  \begin{center}
    {\vspace{-1.5cm}\Large \textbf{Sección \thesection}\vspace{0.5cm}}
    \begin{beamercolorbox}[
      sep=8pt,center]{part title}
      \usebeamerfont{part title}
      \textbf{\insertsection}
    \end{beamercolorbox}
  \end{center}
\end{frame}


\begin{frame}
	\frametitle{\textbf{Design of FFT architecture via folding transformation}}
	\framesubtitle{\secname : \subsecname}
	
	\begin{block}{\centering \textbf{Folding Set}}
		\begin{itemize}\justifying\footnotesize
        	\item Is an ordered set of operations executed by the same functional unit.
        	\item Each folding set contains K entries, where K is called the folding factor.
        	\item The operation in the \textit{j}th position (where goes from 0 to K-1) is called the folding order.
       	\end{itemize}
	\end{block}

	\begin{block}{\centering \textbf{Folding Equations}}
		\begin{itemize}\justifying\footnotesize
			\item Consider an edge $e$ connecting the nodes \textit{U} and \textit{V} with $w(e)$ delays. 
			\item The executions of the \textit{l}th iteration of \textit{U} and \textit{V} are scheduled at the time units $Kl+u$ and $Kl+v$ respectively, where $u$ and $v$ are the folding orders of the nodes \textit{U} and \textit{V}.
			\item  The folding equation for the edge $e$ is:
			\begin{equation}\label{eqn:fold_equation}
				D_F(U \to V) = Kw(e)-P_U+v-u
			\end{equation}
			where $P_U$ is the number of pipeline stages in the node \textit{U}.
	    \end{itemize}
	\end{block}
\end{frame}


\begin{frame}
	\frametitle{\textbf{Design of FFT architecture via folding transformation}}
	\framesubtitle{\secname : \subsecname}
		\vspace{-0.5cm}
	    \begin{columns}[t,onlytextwidth]
	      \begin{column}{0.45\linewidth}
			\begin{block}{\centering \textbf{Folding equations without retiming/pipeline}}	      	
	        	\begin{itemize} \justifying\footnotesize
					\item Consider the folding sets:  \vfill
					\scalebox{0.65}{
					\label{eq:foldingset_16}		 
		 			\parbox{\linewidth}{ 
					\begin{align*}
						A&= \{ A0,A2,A4,A6 \}  & A'&= \{ A1,A3,A5,A7 \} \\
						B&=\{ B1,B3,B0,B2 \}   &B'&=\{ B5,B7,B4,B6 \} 	\\
						C&=\{ C2,C1,C3,C0 \}   &C'&=\{ C6,C5,C7,C4 \} 	\\ 
						D&=\{ D3,D0,D2,D1 \}   &D'&=\{ D7,D4,D6,D5 \}  
					\end{align*}}}
				
					\item For example: \vfill
					\scalebox{0.8}{
		 			\parbox{\linewidth}{ 				
					\begin{align*}
						D_F(D3\to B3)&= v - u \\
								 &= 0 - 1 \\  
								 &= -1
					\end{align*}}}
					\item The folding equations can be derived for all edges. \vfill
	      		\end{itemize}
			\end{block}	      
	      \end{column}
	      \begin{column}{0.50\linewidth}
	      	      \vspace{1cm}
			    \begin{figure}[h!] \centering
	    			\includegraphics[width=0.45\paperwidth]{./image/16points_dfg.png}
	    			\caption{\footnotesize DFG of a radix-$2^3$ 16-point DIF DFT.}
	    		\end{figure}    
	      \end{column}
	    \end{columns}
\end{frame}




\begin{frame}
	\frametitle{\textbf{Design of FFT architecture via folding transformation}}
	\framesubtitle{\secname : \subsecname}
		\vspace{-0.5cm}
	    \begin{columns}[t,onlytextwidth]
	      \begin{column}{0.45\linewidth}
			\begin{block}{\centering \textbf{Folding equations with retimming/pipeline}}	      	
	        	\begin{itemize} \justifying\footnotesize
					\item For the folded system to be realizable, $D_F(U\to V)\geq0$ must hold for all the edges. \vfill
					\item For example: \vfill
						\scalebox{0.8}{
		 				\parbox{\linewidth}{ 				
						\begin{align*}
							D_F(D3\to B3)&= Kw(e) + v - u \\
								 &= 4(1) + 0 - 1 \\  
								 &= 3
						\end{align*}}}
						
					\item This result $D_F(U\to V)\geq0$ for all the edges.
					\item Applying the folding equations for all the edges, the number of registers required is 80. \vfill
	      		\end{itemize}
			\end{block}	      
	      \end{column}
	      \begin{column}{0.50\linewidth}
	      \vspace{1cm}
			    \begin{figure}[h!] \centering
	    			\includegraphics[width=0.45\paperwidth]{./image/16points_dfg_ret.png}
	    			\caption{\footnotesize DFG of a radix-$2^3$ 16-point DIF DFT applying pipeline and retiming.}
	    			\label{fig:16_point_ret}
	    		\end{figure}    
	      \end{column}
	    \end{columns}
\end{frame}

\begin{frame}
	\frametitle{\textbf{Design of FFT architecture via folding transformation}}
	\framesubtitle{\secname : \subsecname}
	\vspace{-0.5cm}
	 \begin{columns}[t,onlytextwidth]
	      \begin{column}{0.5\linewidth}
			\begin{block}{\centering \textbf{Lifetime analysis}}
				\begin{itemize}\justifying\footnotesize
					\item Is a procedure used to compute the minimun number of registers. \vfill
					\item For example, the variable $y_1$ be live during time units $n \in \{1,2,3,4\}$.
					\item The number of live variables $y_i$ during the time units $\{1,2,3,4,5\}$ is $\{4,8,8,8,4\}$. So the number of register for this stage is: \vfill
					\begin{equation*}
						max\{4,8,8,8,4\} = 8
					\end{equation*}
					\item The total number of registers is reduced from 80 to 20. \vfill
		       	\end{itemize}
			\end{block}
   		  \end{column}
   		  \begin{column}{0.5\linewidth}
   		  	 \vspace{-0.5cm}
   			\begin{figure}[h!] \centering
	    		\includegraphics[width=0.40\paperwidth]{./image/life_chart_a.png}
	    		\caption{\footnotesize Lifetime chart for stage 1.}
	    	\end{figure}
	    	\vspace{-1cm}
	    	\begin{figure}[h!] \centering
	    		\includegraphics[width=0.40\paperwidth]{./image/life_chart_b.png}
	    		\caption{\footnotesize Lifetime chart for stage 2.}
	    	\end{figure}
			\vspace{-1cm}	    	
	    	\begin{figure}[h!] \centering
	    		\includegraphics[width=0.40\paperwidth]{./image/life_chart_c.png}
	    		\caption{\footnotesize Lifetime chart for stage 3.}	    	
	    	\end{figure}  
   		\end{column}
	\end{columns}
\end{frame}

\begin{frame}
	\frametitle{\textbf{Design of FFT architecture via folding transformation}}
	\framesubtitle{\secname : \subsecname}
	\vspace{-0.5cm}
	 \begin{columns}[t,onlytextwidth]
	      \begin{column}{0.5\linewidth}
			\begin{block}{\centering \textbf{Forward Register Allcation}}
				\begin{itemize}\justifying\footnotesize
		        	\item This dictates how the variables are assigned to the minimum numbers of registers. \vfill
		        	\item If $R_i$ holds holds a variable in the current ctcle, then $R_{i+1}$ hold the same variable on the next cycle.
		       	\end{itemize}
			\end{block}
			\vspace{-0.3cm}
			\begin{figure}[h!] \centering
	    		\includegraphics[height=0.30\paperheight]{./image/tab-life-a.png}
	    		\caption{\footnotesize Allocation table for stage 1.}
	    	\end{figure}
   		  \end{column}
   		  \begin{column}{0.45\linewidth}
   			\begin{figure}[h!] \centering
	    		\includegraphics[height=0.25\paperheight]{./image/tab-life-b.png}
	    		\caption{\footnotesize Allocation table for stage 2.}
	    	\end{figure}
	    	\vspace{-0.75cm}
	    	\begin{figure}[h!] \centering
	    		\includegraphics[height=0.30\paperheight]{./image/tab-life-c.png}
	    		\caption{\footnotesize Allocation table for stage 3.}
	    	\end{figure}  
   		\end{column}
	\end{columns}
\end{frame}

